\documentclass[10pt,a4paper]{article}
\usepackage[a4paper, total={6in, 9in}]{geometry}
\usepackage[latin1]{inputenc}
\usepackage{amsmath}
\usepackage{amsfonts}
\usepackage{amssymb}
\usepackage{mathtools}
\usepackage{graphicx}
\usepackage{float}
\usepackage[usenames, dvipsnames]{color}
\usepackage{fancyhdr}
\usepackage{enumitem}
\usepackage{listings}
\usepackage[nocomma]{optidef}
\setlength{\headheight}{15.2pt}
\pagestyle{fancy}
\setlength{\parindent}{0em}
\setlength{\parskip}{10pt}
\setlist[enumerate,1]{label=\alph*)}
\setlist[enumerate,2]{label=(\roman*)}
\definecolor{LGray}{gray}{0.9}
\definecolor{DGray}{gray}{0.4}
\usepackage{inconsolata}
\usepackage[T1]{fontenc}
\lstset{language=Matlab, tabsize=2, breaklines=true, postbreak=\mbox{\textcolor{red}{$\hookrightarrow$}\space}, basicstyle=\linespread{0.7}\ttfamily, commentstyle=\color{DGray}} 
\newcommand*\diff{\mathop{}\!\mathrm{d}}



\begin{document}
\title{ENGSCI761 A2}
\author{Benjamin Yi}
\rhead{Benjamin Yi - byi649 - 925302651}
\lhead{ENGSCI761 A2}
	
\section*{Question 1}
\begin{enumerate}
\item
Let \(u_{jk}\) be 1 if item type \(k\) exists in bin \(j\) and 0 otherwise. Let \(I_k\) be the set of items with type \(k\).

We then add two new constraints:

\begin{align}
	\text{M}u_{jk} &\geq \sum_{i \in I_k} y_{ij} &&\forall j, k \\
	\sum_k u_{jk} & \geq 1 &&\forall j \\
	&u_{jk} \quad \text{binary}
\end{align}

where M is some sufficiently large number. (1) enforces the definition of \(u_{jk}\). Because \(\sum_k(u_{jk}) - 1\) counts the number of different types of items in bin \(j\), (2) prevents empty bins from affecting the objective function.

The second objective function is then:
\begin{equation*}
	\text{C}_2 = \sum_j \bigg(\bigg(\sum_k u_{jk}\bigg)-1\bigg)
\end{equation*}

\item 
\(\mathcal{W} = 12\), \(\mathcal{T} = 3\), \(\lambda = 0.01\)

Wastage is integer, so the smallest distance between wastage objective values is 1.\\
The highest possible value for \(\mathcal{T}\) is (number of bins) * (number of item types - 1), which corresponds to the allocation where all bins contains all item types. For problem data 1, this is \(10 * 4 = 40\) as max(number of bins) = (number of items). The lowest possible value is 0, which corresponds to the allocation where one bin contains all items. This means the greatest difference in \(\mathcal{T}\) is 40.

Therefore it is sufficient (but not necessary) that \(1 > \lambda * 40 \rightarrow \lambda < \frac{1}{40}\) so the first objective will always be a priority over the second objective. \(\lambda = 0.01\) was chosen as it fulfills the above condition and is also not so small as to induce numerical error.

\item 
The epsilon-constraint method was used to solve this bi-objective integer problem as it will find all efficient solutions, even the non-supported ones. I assume the set of efficient solutions is not very large - that it can be enumerated within a reasonable time-frame.

The IP for minimising wastage only is solved, and \(\mathcal{T}\) is recorded. \(\mathcal{T}\) is then constrained to be less than the previous solution's \(\mathcal{T}\) and the IP is re-solved. This repeats until \(\mathcal{T}\) is negative or the problem becomes infeasible. The solutions found make up the set of efficient solutions for the bi-objective problem.

The implementation can be run via \lstinline|> python bin_pack_MO_bi.py| with no supporting files needed.

The objective function values for all efficient (not including weakly efficient) solutions are:
\begin{table}[H]
	\centering
\begin{tabular}{|c|c|}
	\hline 
	W & T \\ 
	\hline 
	0 & 7 \\ 
	\hline 
	25 & 3 \\ 
	\hline 
	50 & 2 \\ 
	\hline 
	75 & 1 \\ 
	\hline 
	100 & 0 \\ 
	\hline 
\end{tabular}
\end{table} 
The solution that minimises waste (\(\mathcal{W} = 0\)): 
\begin{table}[H]
\centering
\begin{tabular}{|c|c|}
	\hline 
	Bin & Items \\ 
	\hline 
	1 & 3, 15, 16 \\ 
	\hline 
	2 & 6, 8, 9, 12 \\ 
	\hline 
	3 & 2, 7, 11 \\ 
	\hline 
	4 & 1, 20 \\ 
	\hline 
	5 & 5, 10, 17, 18 \\ 
	\hline 
	7 & 4, 13, 14, 19\\ 
	\hline 
\end{tabular}
\end{table}
The next solution (\(\mathcal{W} = 25\)):
\begin{table}[H]
	\centering
\begin{tabular}{|c|c|}
	\hline 
	Bin & Items \\ 
	\hline 
	1 & 2, 9, 10, 12, 18 \\ 
	\hline 
	2 & 3, 5, 13, 14 \\ 
	\hline 
	3 & 1, 6 \\ 
	\hline 
	4 & 11, 19, 20 \\ 
	\hline 
	5 & 15, 16 \\ 
	\hline 
	6 & 4, 8, 17 \\ 
	\hline 
	7 & 7 \\ 
	\hline 
\end{tabular} 
\end{table}

\item 
Introduce another objective \(\text{C}_3\) that minimises the maximum number of different types of items in each bin.

This is done by adding a new variable:
\begin{equation*}
	m \geq \sum_k u_{jk} \quad \forall j
\end{equation*}
with objective function:
\begin{equation*}
\text{C}_3 = \text{minimise } m
\end{equation*}
\end{enumerate}
\end{document}

